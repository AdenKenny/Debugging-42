\chapter{Future Work}\label{C:con}
A significant amount of work has been accomplished so far in finding bugs in the 42 compiler. Although the quantity of bugs found is perhaps not as high as hoped, but the quality of bugs found, if all factors are considered, is high. One of the mains reason the quantity of bugs found is quite low is due to the fact that no automated testing tools have been thus far. Which means that the overall quantity of tests written so far is not very large. All tests have been written manually, and the majority of these manually written tests have been fairly general programs which have been written in a black-box testing style. These black-box tests did not take into account the compiler code or previous knowledge of the compiler. A minority of programs were written in a grey-box testing style, and these programs proved to be the most effective at finding bugs. They had a probability of 9.2\% of finding a bug compared to the black-box style tests which had a probability of 3.1\%. We can therefore conclude that the grey-box testing style was more effective when contrasted to the black-box testing style. A conclusion that can be drawn from these results is that a focus should probably be placed upon a grey-box testing style. It must be noted that this would not exclude the possibility of black-box tests being written in the future, as they could still be useful and black-box testing has found bugs so far.

~\\
The next step for this project is to incorporate automated testing tools such as fuzzers. These tools may potentially uncover a large amount of bugs by virtue of allowing far more tests to be created than would be possible with purely manually creating tests. It is a possibility that automated test may not be as effective as hoped in this scenario due to various factors discussed in Section 2.4.2 of this report. The possibility of this will be mitigated by still writing manual tests. Automated testing will take place in parallel with manual testing. The future manual testing will almost certainly take on a more grey-box testing strategy, as so far, it has proven to be both more effective at finding bugs on the whole, and more effective per test written. 

~\\
Overall, the future work on this project will consist of more testing, specifically targeted tests with knowledge of the code of the compiler, as well as the use of automated testing tools and techniques \cite{Miller:1995} \cite{javaAfl} \cite{javaFuzz} \cite{Chen:2017} \cite{Chen:2016} to increase the quantity of tests produced. The combination of high quality targeted testing, and more general automated tests will provide a significantly more rounded testing strategy than has been used so far over the course of this project.

